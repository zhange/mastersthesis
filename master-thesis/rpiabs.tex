%%%%%%%%%%%%%%%%%%%%%%%%%%%%%%%%%%%%%%%%%%%%%%%%%%%%%%%%%%%%%%%%%%%
%                                                                 %
%                            ABSTRACT                             %
%                                                                 %
%%%%%%%%%%%%%%%%%%%%%%%%%%%%%%%%%%%%%%%%%%%%%%%%%%%%%%%%%%%%%%%%%%%
\specialhead{ABSTRACT}

The use of sketching amongst architects contributes  significant benefits in the creation of creative and professional projects. Sketching offers a simple method of conveying ideas to an audience, without a large time commitment. As a result, architects can prototype designs rapidly, exploring many patterns while receiving feedback on their work. Sketching influences the design process by allowing the designer to work more creatively and effectively through quick, numerous iterations. The use of sketching in architectural design interfaces has only increased the flexibility and possible designs for users. However, variation in human input increases the difficulty to accurately recognize the intent of the user. Traditional methods of interacting with architectural design interfaces rely on creation of fixed sized primitives. When interacting with such systems, design options feel more limited. \\

I present an alternative sketch-based interface for the Online Architectural Sketching Interface for Simulations. This interface emulates drawing using paper and pencil, allowing the user to create lines on a canvas. Using the user input, the interface recognizes certain series of strokes as objects, creating furniture items as feedback to the user. The strokes are also processed by the system as different gestures, allowing the user a multitude of options to edit designs. \\

%Using these shapes, the system classifies them as possible furniture items.

Contributions of this thesis include the development of an architectural sketching interface, algorithms to perform recognition and fitting of interface primitives, and conduction of a pilot study. 
%%%%%%%%%%%%%%%%%%%%%%%%%%%%%%%%%%%%%%%%%%%%%%%%%%%%%
